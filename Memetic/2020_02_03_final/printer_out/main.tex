%!TEX program = lualatex
\documentclass[12pt,a4paper,twoside]{article}

%----------------------------------------
% Packages
%----------------------------------------

% Programming facilities
\usepackage{etoolbox}
\usepackage{ifxetex}
\usepackage{ifluatex}
% Encoding
\usepackage[T1]{fontenc}
\ifboolexpr{bool{xetex} or bool{luatex}}{%
	\usepackage{fontspec}
}{%
	\usepackage[utf8]{inputenc}
}
% General purpose
\usepackage[english]{babel}
\usepackage[top=2cm, bottom=1.8cm, left=1.8cm, right=1.8cm, head=14pt, foot=36pt]{geometry}
\usepackage{lmodern}
\usepackage{hyperref}
\usepackage{fancyhdr}
\usepackage[table]{xcolor}
\usepackage{float}
\usepackage{datetime2}
\usepackage{chngcntr}
\usepackage{pdflscape}
\usepackage{ifthen}
% Mathematics
\usepackage{amsmath}
\usepackage{amssymb}
\usepackage{mathrsfs}
\usepackage{amsthm}
\usepackage{dsfont}
\usepackage{braket}
\usepackage{stmaryrd}
% Tables
\usepackage{array}
\usepackage{tabularx}
\usepackage{longtable}
\usepackage{ltablex}
\usepackage{tabu}
\usepackage{booktabs}
\usepackage{multirow}
% PGF-TikZ
\usepackage{pgf}
\usepackage{pgfplots}
\pgfplotsset{compat=1.16}
\usepackage{tikz}
\usepackage[mode=tex]{standalone}
\usepackage{import}

%----------------------------------------
% Configuration
%----------------------------------------

\DTMsavetimestamp{generated}{2020-02-08T14:00:18Z}

%----------------------------------------
% Informations
%----------------------------------------

% Cover infos
\title{USCP results}
\author{\url{https://github.com/pinam45/USCP}}
\date{\DTMusedate{generated} \DTMusetime{generated}}

% Fancy style options
\lhead{\small USCP results}
\rhead{\small \DTMusedate{generated} \DTMusetime{generated}}
\chead{}
\lfoot{\url{https://github.com/pinam45/USCP}}
\rfoot{\thepage}
\cfoot{}
\pagestyle{fancy}

%----------------------------------------
% Tables
%----------------------------------------

\newcolumntype{L}[1]{>{\raggedright\let\newline\\\arraybackslash\hspace{0pt}}m{#1}}
\newcolumntype{C}[1]{>{\centering\let\newline\\\arraybackslash\hspace{0pt}}m{#1}}
\newcolumntype{R}[1]{>{\raggedleft\let\newline\\\arraybackslash\hspace{0pt}}m{#1}}
\newcolumntype{Y}{>{\centering\arraybackslash}X}
\counterwithin{table}{section}

\newcommand{\tableplot}[1]{%
	\raisebox{-2pt}{%
		\resizebox{25pt}{9pt}{%
			\begin{tikzpicture}%
				\begin{axis}[%
					ybar,%
					width=1000pt,%
					%height=55pt,%
			  		bar width=0.75,%
					xmin=0,%
					ymin=0,%
					enlargelimits=0,%
			  		axis lines=none,%
			  		xtick=\empty,%
			  		ytick=\empty,%
				]%
				\foreach \x [count=\xi] in {#1}
				{
					\ifthenelse{\isodd\xi}{
						\addplot+[
							black,
							fill=black,
							bar shift=-0.5
						] coordinates{(\xi,\x)};
					}{
						\addplot+[
							gray,
							fill=gray,
							bar shift=-0.5
						] coordinates{(\xi,\x)};
					}
				}
				\end{axis}
			\end{tikzpicture}%
		}%
	}%
}

%----------------------------------------
% Plots
%----------------------------------------

\pgfplotsset{
	compat=1.16,
	table/col sep=comma
}
\pgfplotsset{
	table/search path={./plots},
}
\usetikzlibrary{intersections}
\usepgfplotslibrary{fillbetween}

%----------------------------------------
% Links
%----------------------------------------

% Links config, especialy for the table of contents
\hypersetup{
  colorlinks=true,
  linkcolor=black,
  urlcolor=blue,
  linktoc=all,
  citecolor=black
}

%----------------------------------------
% Document
%----------------------------------------
\begin{document}
	\maketitle{}
	\tableofcontents{}
	\listoftables{}
	\newpage\section{Instances}
		%!TEX root = ../main.tex
\begin{longtable}{cccccc}
	\caption{OR-Library base instances}\\
	\toprule
	Set & Instances & Rows & Columns & Density & Cost range\\
	\midrule
	\endfirsthead
	\caption{OR-Library base instances (continued)}\\
	\toprule
	Set & Instances & Rows & Columns & Density & Cost range\\
	\midrule
	\endhead
	\bottomrule
	\endfoot
	4 & 10 & 200 & 1000 & 2\% & [1;100] \\
	5 & 10 & 200 & 2000 & 2\% & [1;100] \\
	6 & 5 & 200 & 1000 & 5\% & [1;100] \\
	A & 5 & 300 & 3000 & 2\% & [1;100] \\
	B & 5 & 300 & 3000 & 5\% & [1;100] \\
	C & 5 & 400 & 4000 & 2\% & [1;100] \\
	D & 5 & 400 & 4000 & 5\% & [1;100] \\
	E & 5 & 50 & 500 & 20\% & [1;1] \\
	NRE & 5 & 500 & 5000 & 10\% & [1;100] \\
	NRF & 5 & 500 & 5000 & 20\% & [1;100] \\
	NRG & 5 & 1000 & 10000 & 2\% & [1;100] \\
	NRH & 5 & 1000 & 10000 & 5\% & [1;100] \\
\end{longtable}

		%!TEX root = ../main.tex
\begin{longtable}{ccccc}
	\caption{OR-Library CYC CLR instances}\\
	\toprule
	Instance & Rows & Columns & Density & Cost range\\
	\midrule
	\endfirsthead
	\caption{OR-Library CYC CLR instances (continued)}\\
	\toprule
	Instance & Rows & Columns & Density & Cost range\\
	\midrule
	\endhead
	\bottomrule
	\endfoot
	CYC6 & 240 & 192 & 2.1\% & [1;1]\\
	CYC7 & 672 & 448 & 0.9\% & [1;1]\\
	CYC8 & 1792 & 1024 & 0.4\% & [1;1]\\
	CYC9 & 4608 & 2304 & 0.2\% & [1;1]\\
	CYC10 & 11520 & 5120 & 0.1\% & [1;1]\\
	CYC11 & 28160 & 11264 & 0.04\% & [1;1]\\
	CLR10 & 511 & 210 & 12\% & [1;1]\\
	CLR11 & 1023 & 330 & 12\% & [1;1]\\
	CLR12 & 2047 & 495 & 12\% & [1;1]\\
	CLR13 & 4095 & 715 & 12\% & [1;1]\\
\end{longtable}

		%!TEX root = ../main.tex
\begin{longtable}{ccccc}
	\caption{OR-Library RAIL instances}\\
	\toprule
	Instance & Rows & Columns & Density & Cost range\\
	\midrule
	\endfirsthead
	\caption{OR-Library RAIL instances (continued)}\\
	\toprule
	Instance & Rows & Columns & Density & Cost range\\
	\midrule
	\endhead
	\bottomrule
	\endfoot
	RAIL507 & 507 & 63009 & 1.3\% & [1;2]\\
	RAIL516 & 516 & 47311 & 1.3\% & [1;2]\\
	RAIL582 & 582 & 55515 & 1.2\% & [1;2]\\
	RAIL2536 & 2536 & 1081841 & 0.4\% & [1;2]\\
	RAIL2586 & 2586 & 920683 & 0.3\% & [1;2]\\
	RAIL4284 & 4284 & 1092610 & 0.2\% & [1;2]\\
	RAIL4872 & 4872 & 968672 & 0.2\% & [1;2]\\
\end{longtable}

		%!TEX root = ../main.tex
\begin{longtable}{ccccc}
	\caption{STS instances}\\
	\toprule
	Instance & Rows & Columns & Density & Cost range\\
	\midrule
	\endfirsthead
	\caption{STS instances (continued)}\\
	\toprule
	Instance & Rows & Columns & Density & Cost range\\
	\midrule
	\endhead
	\bottomrule
	\endfoot
	STS9 & 12 & 9 & 33.3\% & [1;1]\\
	STS15 & 35 & 15 & 20\% & [1;1]\\
	STS27 & 117 & 27 & 11.1\% & [1;1]\\
	STS45 & 330 & 45 & 6.7\% & [1;1]\\
	STS81 & 1080 & 81 & 3.7\% & [1;1]\\
	STS135 & 3015 & 135 & 2.2\% & [1;1]\\
	STS243 & 9801 & 243 & 1.2\% & [1;1]\\
	STS405 & 27270 & 405 & 0.7\% & [1;1]\\
	STS729 & 88452 & 729 & 0.4\% & [1;1]\\
	STS1215 & 245835 & 1215 & 0.2\% & [1;1]\\
	STS2187 & 796797 & 2187 & 0.1\% & [1;1]\\
\end{longtable}

	\begin{landscape}
		\newpage\section{Results}
			%!TEX root = ../main.tex
\rowcolors{1}{gray!10}{white}
\footnotesize
\begin{longtable}{@{\extracolsep{0pt}}cc{c}cr{c}ccrcrr{c}ccrrcrr}
	\hiderowcolors
	\caption{Results}\\
	\toprule
	\multirow{2}{*}{\raisebox{-\heavyrulewidth}{Inst.}} & \multirow{2}{*}{\raisebox{-\heavyrulewidth}{BKS}} && \multicolumn{2}{l}{Greedy} && \multicolumn{6}{l}{RWLS} && \multicolumn{7}{l}{Memetic(RWLS)}\\
	\cmidrule{4-5}\cmidrule{7-12}\cmidrule{14-20}
	 & && Val. & \multicolumn{1}{c}{T(s)} && Best & Av. & \multicolumn{1}{c}{\#Best} & Dist. & \multicolumn{1}{c}{Steps} & \multicolumn{1}{c}{T(s)} && Best & Av. & \multicolumn{1}{c}{\#Best} & Dist. & \multicolumn{1}{c}{Gen.} & \multicolumn{1}{c}{Steps} & \multicolumn{1}{c}{T(s)}\\
	\midrule
	\endfirsthead
	\caption[]{Results (continued)}\\
	\toprule
	\multirow{2}{*}{\raisebox{-\heavyrulewidth}{Inst.}} & \multirow{2}{*}{\raisebox{-\heavyrulewidth}{BKS}} && \multicolumn{2}{l}{Greedy} && \multicolumn{6}{l}{RWLS} && \multicolumn{7}{l}{Memetic(RWLS)}\\
	\cmidrule{4-5}\cmidrule{7-12}\cmidrule{14-20}
	 & && Val. & \multicolumn{1}{c}{T(s)} && Best & Av. & \multicolumn{1}{c}{\#Best} & Dist. & \multicolumn{1}{c}{Steps} & \multicolumn{1}{c}{T(s)} && Best & Av. & \multicolumn{1}{c}{\#Best} & Dist. & \multicolumn{1}{c}{Gen.} & \multicolumn{1}{c}{Steps} & \multicolumn{1}{c}{T(s)}\\
	\midrule
	\endhead
	\bottomrule
	\endfoot
	\showrowcolors
	4.1
	&
	38
	&&
	41
	&
	0.00
	&&
	\textbf{38}
	&
	38.0
	&
	100/100
	&
	&
	12 273
	&
	0.04
	&&
	\textbf{38}
	&
	38.0
	&
	100/100
	&
	&
	3.0
	&
	17 511
	&
	0.27
	\\
	4.2
	&
	37
	&&
	41
	&
	0.00
	&&
	\textbf{37}
	&
	37.0
	&
	100/100
	&
	&
	2 211
	&
	0.01
	&&
	\textbf{37}
	&
	37.0
	&
	100/100
	&
	&
	0.6
	&
	2 464
	&
	0.07
	\\
	4.3
	&
	38
	&&
	43
	&
	0.00
	&&
	\textbf{38}
	&
	38.0
	&
	100/100
	&
	&
	3 146
	&
	0.02
	&&
	\textbf{38}
	&
	38.0
	&
	100/100
	&
	&
	0.8
	&
	3 285
	&
	0.11
	\\
	4.4
	&
	38
	&&
	44
	&
	0.00
	&&
	\textbf{38}
	&
	38.0
	&
	100/100
	&
	&
	80 979
	&
	0.12
	&&
	\textbf{38}
	&
	38.0
	&
	100/100
	&
	&
	9.8
	&
	93 265
	&
	0.95
	\\
	4.5
	&
	38
	&&
	44
	&
	0.00
	&&
	\textbf{38}
	&
	38.0
	&
	100/100
	&
	&
	11 631
	&
	0.03
	&&
	\textbf{38}
	&
	38.0
	&
	100/100
	&
	&
	3.2
	&
	18 739
	&
	0.48
	\\
	4.6
	&
	37
	&&
	43
	&
	0.00
	&&
	\textbf{37}
	&
	37.0
	&
	100/100
	&
	&
	52 282
	&
	0.08
	&&
	\textbf{37}
	&
	37.0
	&
	100/100
	&
	&
	7.8
	&
	61 395
	&
	0.99
	\\
	4.7
	&
	38
	&&
	43
	&
	0.00
	&&
	\textbf{38}
	&
	38.0
	&
	100/100
	&
	&
	28 237
	&
	0.06
	&&
	\textbf{38}
	&
	38.0
	&
	100/100
	&
	&
	5.4
	&
	38 385
	&
	0.82
	\\
	4.8
	&
	37
	&&
	42
	&
	0.00
	&&
	\textbf{37}
	&
	37.0
	&
	100/100
	&
	&
	43 596
	&
	0.08
	&&
	\textbf{37}
	&
	37.0
	&
	100/100
	&
	&
	6.8
	&
	55 101
	&
	0.84
	\\
	4.9
	&
	38
	&&
	42
	&
	0.00
	&&
	\textbf{38}
	&
	38.0
	&
	100/100
	&
	&
	11 189
	&
	0.03
	&&
	\textbf{38}
	&
	38.0
	&
	100/100
	&
	&
	2.9
	&
	16 328
	&
	0.27
	\\
	4.10
	&
	38
	&&
	43
	&
	0.00
	&&
	\textbf{38}
	&
	38.0
	&
	100/100
	&
	&
	51 068
	&
	0.09
	&&
	\textbf{38}
	&
	38.0
	&
	100/100
	&
	&
	8.1
	&
	68 515
	&
	1.47
	\\
	5.1
	&
	34
	&&
	37
	&
	0.00
	&&
	\textbf{34}
	&
	34.0
	&
	100/100
	&
	&
	119 262
	&
	0.21
	&&
	\textbf{34}
	&
	34.0
	&
	100/100
	&
	&
	9.1
	&
	143 032
	&
	1.18
	\\
	5.2
	&
	34
	&&
	38
	&
	0.00
	&&
	\textbf{34}
	&
	34.0
	&
	100/100
	&
	&
	43 279
	&
	0.09
	&&
	\textbf{34}
	&
	34.0
	&
	100/100
	&
	&
	5.5
	&
	69 185
	&
	0.62
	\\
	5.3
	&
	34
	&&
	37
	&
	0.00
	&&
	\textbf{34}
	&
	34.0
	&
	100/100
	&
	&
	10 819
	&
	0.04
	&&
	\textbf{34}
	&
	34.0
	&
	100/100
	&
	&
	1.7
	&
	14 504
	&
	0.14
	\\
	5.4
	&
	34
	&&
	39
	&
	0.00
	&&
	\textbf{34}
	&
	34.0
	&
	100/100
	&
	&
	17 091
	&
	0.05
	&&
	\textbf{34}
	&
	34.0
	&
	100/100
	&
	&
	2.6
	&
	26 119
	&
	0.38
	\\
	5.5
	&
	34
	&&
	37
	&
	0.00
	&&
	\textbf{34}
	&
	34.0
	&
	100/100
	&
	&
	14 079
	&
	0.04
	&&
	\textbf{34}
	&
	34.0
	&
	100/100
	&
	&
	2.1
	&
	20 162
	&
	0.25
	\\
	5.6
	&
	34
	&&
	40
	&
	0.00
	&&
	\textbf{34}
	&
	34.0
	&
	100/100
	&
	&
	32 226
	&
	0.09
	&&
	\textbf{34}
	&
	34.0
	&
	100/100
	&
	&
	3.6
	&
	41 484
	&
	0.32
	\\
	5.7
	&
	34
	&&
	38
	&
	0.00
	&&
	\textbf{34}
	&
	34.0
	&
	100/100
	&
	&
	8 136
	&
	0.03
	&&
	\textbf{34}
	&
	34.0
	&
	100/100
	&
	&
	1.1
	&
	9 289
	&
	0.13
	\\
	5.8
	&
	34
	&&
	39
	&
	0.00
	&&
	\textbf{34}
	&
	34.0
	&
	100/100
	&
	&
	40 656
	&
	0.09
	&&
	\textbf{34}
	&
	34.0
	&
	100/100
	&
	&
	3.8
	&
	46 375
	&
	0.40
	\\
	5.9
	&
	35
	&&
	38
	&
	0.00
	&&
	\textbf{35}
	&
	35.0
	&
	100/100
	&
	&
	11 414
	&
	0.03
	&&
	\textbf{35}
	&
	35.0
	&
	100/100
	&
	&
	1.6
	&
	14 384
	&
	0.27
	\\
	5.10
	&
	34
	&&
	39
	&
	0.00
	&&
	\textbf{34}
	&
	34.0
	&
	100/100
	&
	&
	55 893
	&
	0.11
	&&
	\textbf{34}
	&
	34.0
	&
	100/100
	&
	&
	7.2
	&
	103 038
	&
	0.73
	\\
	6.1
	&
	21
	&&
	23
	&
	0.00
	&&
	\textbf{21}
	&
	21.0
	&
	100/100
	&
	&
	3 505
	&
	0.02
	&&
	\textbf{21}
	&
	21.0
	&
	100/100
	&
	&
	1.0
	&
	4 369
	&
	0.19
	\\
	6.2
	&
	20
	&&
	22
	&
	0.00
	&&
	\textbf{20}
	&
	20.0
	&
	100/100
	&
	&
	19 049
	&
	0.07
	&&
	\textbf{20}
	&
	20.0
	&
	100/100
	&
	&
	4.0
	&
	23 757
	&
	0.26
	\\
	6.3
	&
	21
	&&
	23
	&
	0.00
	&&
	\textbf{21}
	&
	21.0
	&
	100/100
	&
	&
	3 049
	&
	0.02
	&&
	\textbf{21}
	&
	21.0
	&
	100/100
	&
	&
	0.9
	&
	3 814
	&
	0.09
	\\
	6.4
	&
	20
	&&
	22
	&
	0.00
	&&
	\textbf{20}
	&
	20.0
	&
	100/100
	&
	&
	184 476
	&
	0.47
	&&
	\textbf{20}
	&
	20.0
	&
	100/100
	&
	&
	17.2
	&
	196 377
	&
	1.77
	\\
	6.5
	&
	21
	&&
	23
	&
	0.00
	&&
	\textbf{21}
	&
	21.0
	&
	100/100
	&
	&
	5 122
	&
	0.02
	&&
	\textbf{21}
	&
	21.0
	&
	100/100
	&
	&
	1.5
	&
	6 991
	&
	0.15
	\\
	A.1
	&
	38
	&&
	42
	&
	0.01
	&&
	\textbf{38}
	&
	38.6
	&
	43/100
	&
	\tableplot{43,57}
	&
	12 761 830
	&
	30.93
	&&
	\textbf{38}
	&
	38.2
	&
	77/100
	&
	\tableplot{77,23}
	&
	87.5
	&
	9 407 003
	&
	22.42
	\\
	A.2
	&
	38
	&&
	42
	&
	0.00
	&&
	\textbf{38}
	&
	38.0
	&
	100/100
	&
	&
	615 151
	&
	1.58
	&&
	\textbf{38}
	&
	38.0
	&
	100/100
	&
	&
	21.5
	&
	914 369
	&
	2.90
	\\
	A.3
	&
	38
	&&
	43
	&
	0.00
	&&
	\textbf{38}
	&
	38.2
	&
	76/100
	&
	\tableplot{76,24}
	&
	10 159 377
	&
	24.48
	&&
	\textbf{38}
	&
	38.1
	&
	94/100
	&
	\tableplot{94,6}
	&
	98.1
	&
	9 677 671
	&
	22.09
	\\
	A.4
	&
	37
	&&
	41
	&
	0.00
	&&
	\textbf{37}
	&
	37.0
	&
	100/100
	&
	&
	2 029 071
	&
	5.22
	&&
	\textbf{37}
	&
	37.0
	&
	100/100
	&
	&
	34.0
	&
	1 913 294
	&
	5.95
	\\
	A.5
	&
	38
	&&
	43
	&
	0.00
	&&
	\textbf{38}
	&
	38.0
	&
	100/100
	&
	&
	95 806
	&
	0.28
	&&
	\textbf{38}
	&
	38.0
	&
	100/100
	&
	&
	7.5
	&
	162 586
	&
	0.96
	\\
	B.1
	&
	22
	&&
	24
	&
	0.00
	&&
	\textbf{22}
	&
	22.0
	&
	100/100
	&
	&
	13 789
	&
	0.12
	&&
	\textbf{22}
	&
	22.0
	&
	100/100
	&
	&
	1.7
	&
	22 586
	&
	0.45
	\\
	B.2
	&
	22
	&&
	23
	&
	0.00
	&&
	\textbf{22}
	&
	22.0
	&
	100/100
	&
	&
	13 408
	&
	0.13
	&&
	\textbf{22}
	&
	22.0
	&
	100/100
	&
	&
	1.4
	&
	17 636
	&
	0.54
	\\
	B.3
	&
	22
	&&
	23
	&
	0.00
	&&
	\textbf{22}
	&
	22.0
	&
	100/100
	&
	&
	30 972
	&
	0.27
	&&
	\textbf{22}
	&
	22.0
	&
	100/100
	&
	&
	3.3
	&
	53 277
	&
	0.78
	\\
	B.4
	&
	22
	&&
	24
	&
	0.00
	&&
	\textbf{22}
	&
	22.0
	&
	100/100
	&
	&
	40 097
	&
	0.33
	&&
	\textbf{22}
	&
	22.0
	&
	100/100
	&
	&
	4.0
	&
	65 972
	&
	0.89
	\\
	B.5
	&
	22
	&&
	25
	&
	0.00
	&&
	\textbf{22}
	&
	22.0
	&
	100/100
	&
	&
	24 728
	&
	0.21
	&&
	\textbf{22}
	&
	22.0
	&
	100/100
	&
	&
	3.1
	&
	47 279
	&
	0.68
	\\
	C.1
	&
	40
	&&
	47
	&
	0.00
	&&
	43
	&
	43.0
	&
	100/100
	&
	&
	105 957
	&
	0.44
	&&
	43
	&
	43.0
	&
	100/100
	&
	&
	6.5
	&
	180 556
	&
	1.26
	\\
	C.2
	&
	40
	&&
	47
	&
	0.01
	&&
	43
	&
	43.0
	&
	100/100
	&
	&
	144 167
	&
	0.57
	&&
	43
	&
	43.0
	&
	100/100
	&
	&
	7.4
	&
	214 135
	&
	1.25
	\\
	C.3
	&
	40
	&&
	47
	&
	0.00
	&&
	43
	&
	43.0
	&
	100/100
	&
	&
	135 877
	&
	0.53
	&&
	43
	&
	43.0
	&
	100/100
	&
	&
	7.9
	&
	238 576
	&
	1.80
	\\
	C.4
	&
	40
	&&
	46
	&
	0.00
	&&
	43
	&
	43.0
	&
	100/100
	&
	&
	109 387
	&
	0.44
	&&
	43
	&
	43.0
	&
	100/100
	&
	&
	6.9
	&
	190 396
	&
	1.66
	\\
	C.5
	&
	43
	&&
	47
	&
	0.01
	&&
	\textbf{43}
	&
	43.0
	&
	100/100
	&
	&
	321 610
	&
	1.27
	&&
	\textbf{43}
	&
	43.0
	&
	100/100
	&
	&
	13.7
	&
	544 859
	&
	2.89
	\\
	D.1
	&
	24
	&&
	27
	&
	0.00
	&&
	\textbf{24}
	&
	24.0
	&
	100/100
	&
	&
	222 324
	&
	2.72
	&&
	\textbf{24}
	&
	24.0
	&
	100/100
	&
	&
	12.1
	&
	406 788
	&
	4.69
	\\
	D.2
	&
	24
	&&
	26
	&
	0.00
	&&
	\textbf{24}
	&
	24.0
	&
	100/100
	&
	&
	4 672 875
	&
	56.75
	&&
	\textbf{24}
	&
	24.0
	&
	100/100
	&
	&
	64.2
	&
	4 483 439
	&
	36.38
	\\
	D.3
	&
	24
	&&
	27
	&
	0.00
	&&
	\textbf{24}
	&
	24.1
	&
	90/100
	&
	\tableplot{90,10}
	&
	8 561 626
	&
	118.75
	&&
	\textbf{24}
	&
	24.1
	&
	95/100
	&
	\tableplot{95,5}
	&
	129.6
	&
	10 710 484
	&
	87.20
	\\
	D.4
	&
	24
	&&
	26
	&
	0.00
	&&
	\textbf{24}
	&
	24.0
	&
	98/100
	&
	\tableplot{98,2}
	&
	7 923 982
	&
	98.63
	&&
	\textbf{24}
	&
	24.1
	&
	94/100
	&
	\tableplot{94,6}
	&
	95.5
	&
	10 046 106
	&
	84.28
	\\
	D.5
	&
	24
	&&
	27
	&
	0.00
	&&
	\textbf{24}
	&
	24.2
	&
	85/100
	&
	\tableplot{85,15}
	&
	11 600 837
	&
	146.45
	&&
	\textbf{24}
	&
	24.1
	&
	89/100
	&
	\tableplot{89,11}
	&
	100.9
	&
	10 368 043
	&
	79.32
	\\
	E.1
	&
	5
	&&
	\textbf{5}
	&
	0.00
	&&
	\textbf{5}
	&
	5.0
	&
	100/100
	&
	&
	0
	&
	0.00
	&&
	\textbf{5}
	&
	5.0
	&
	100/100
	&
	&
	0.0
	&
	15
	&
	0.04
	\\
	E.2
	&
	5
	&&
	\textbf{5}
	&
	0.00
	&&
	\textbf{5}
	&
	5.0
	&
	100/100
	&
	&
	0
	&
	0.00
	&&
	\textbf{5}
	&
	5.0
	&
	100/100
	&
	&
	0.0
	&
	0
	&
	0.02
	\\
	E.3
	&
	5
	&&
	\textbf{5}
	&
	0.00
	&&
	\textbf{5}
	&
	5.0
	&
	100/100
	&
	&
	0
	&
	0.00
	&&
	\textbf{5}
	&
	5.0
	&
	100/100
	&
	&
	0.0
	&
	0
	&
	0.02
	\\
	E.4
	&
	5
	&&
	6
	&
	0.00
	&&
	\textbf{5}
	&
	5.0
	&
	100/100
	&
	&
	5
	&
	0.01
	&&
	\textbf{5}
	&
	5.0
	&
	100/100
	&
	&
	0.0
	&
	3
	&
	0.04
	\\
	E.5
	&
	5
	&&
	\textbf{5}
	&
	0.00
	&&
	\textbf{5}
	&
	5.0
	&
	100/100
	&
	&
	0
	&
	0.00
	&&
	\textbf{5}
	&
	5.0
	&
	100/100
	&
	&
	0.0
	&
	0
	&
	0.03
	\\
	NRE.1
	&
	16
	&&
	18
	&
	0.00
	&&
	\textbf{16}
	&
	16.9
	&
	13/100
	&
	\tableplot{13,87}
	&
	14 995 822
	&
	810.52
	&&
	\textbf{16}
	&
	16.8
	&
	19/100
	&
	\tableplot{19,81}
	&
	122.7
	&
	16 614 128
	&
	480.72
	\\
	NRE.2
	&
	16
	&&
	18
	&
	0.00
	&&
	\textbf{16}
	&
	16.9
	&
	12/100
	&
	\tableplot{12,88}
	&
	11 867 559
	&
	652.83
	&&
	\textbf{16}
	&
	16.8
	&
	21/100
	&
	\tableplot{21,79}
	&
	111.0
	&
	13 922 163
	&
	466.17
	\\
	NRE.3
	&
	16
	&&
	18
	&
	0.00
	&&
	\textbf{16}
	&
	16.8
	&
	19/100
	&
	\tableplot{19,81}
	&
	17 423 058
	&
	905.52
	&&
	\textbf{16}
	&
	16.8
	&
	18/100
	&
	\tableplot{18,82}
	&
	99.2
	&
	13 062 854
	&
	429.76
	\\
	NRE.4
	&
	16
	&&
	18
	&
	0.00
	&&
	\textbf{16}
	&
	16.8
	&
	24/100
	&
	\tableplot{24,76}
	&
	15 085 446
	&
	817.84
	&&
	\textbf{16}
	&
	16.7
	&
	26/100
	&
	\tableplot{26,74}
	&
	117.2
	&
	15 723 271
	&
	481.29
	\\
	NRE.5
	&
	16
	&&
	18
	&
	0.00
	&&
	\textbf{16}
	&
	16.9
	&
	12/100
	&
	\tableplot{12,88}
	&
	17 958 136
	&
	978.01
	&&
	\textbf{16}
	&
	16.9
	&
	11/100
	&
	\tableplot{11,89}
	&
	136.8
	&
	20 855 715
	&
	629.57
	\\
	NRF.1
	&
	10
	&&
	11
	&
	0.00
	&&
	\textbf{10}
	&
	10.0
	&
	100/100
	&
	&
	42 418
	&
	6.39
	&&
	\textbf{10}
	&
	10.0
	&
	100/100
	&
	&
	2.2
	&
	50 140
	&
	6.66
	\\
	NRF.2
	&
	10
	&&
	11
	&
	0.00
	&&
	\textbf{10}
	&
	10.0
	&
	100/100
	&
	&
	47 572
	&
	6.13
	&&
	\textbf{10}
	&
	10.0
	&
	100/100
	&
	&
	2.1
	&
	46 293
	&
	6.76
	\\
	NRF.3
	&
	10
	&&
	11
	&
	0.00
	&&
	\textbf{10}
	&
	10.0
	&
	100/100
	&
	&
	43 518
	&
	6.07
	&&
	\textbf{10}
	&
	10.0
	&
	100/100
	&
	&
	2.1
	&
	47 704
	&
	7.49
	\\
	NRF.4
	&
	10
	&&
	11
	&
	0.00
	&&
	\textbf{10}
	&
	10.0
	&
	100/100
	&
	&
	44 798
	&
	6.34
	&&
	\textbf{10}
	&
	10.0
	&
	100/100
	&
	&
	2.2
	&
	49 589
	&
	6.27
	\\
	NRF.5
	&
	10
	&&
	11
	&
	0.00
	&&
	\textbf{10}
	&
	10.0
	&
	100/100
	&
	&
	36 524
	&
	4.92
	&&
	\textbf{10}
	&
	10.0
	&
	100/100
	&
	&
	2.3
	&
	52 414
	&
	6.81
	\\
	NRG.1
	&
	61
	&&
	65
	&
	0.01
	&&
	\textbf{60*}
	&
	60.4
	&
	56/100
	&
	\tableplot{56,44}
	&
	16 576 199
	&
	229.14
	&&
	\textbf{60*}
	&
	60.5
	&
	50/100
	&
	\tableplot{50,50}
	&
	40.1
	&
	19 281 058
	&
	144.25
	\\
	NRG.2
	&
	61
	&&
	65
	&
	0.02
	&&
	\textbf{60*}
	&
	60.5
	&
	49/100
	&
	\tableplot{49,51}
	&
	15 700 726
	&
	198.20
	&&
	\textbf{60*}
	&
	60.5
	&
	46/100
	&
	\tableplot{46,54}
	&
	40.8
	&
	19 866 909
	&
	163.48
	\\
	NRG.3
	&
	61
	&&
	66
	&
	0.03
	&&
	\textbf{60*}
	&
	61.0
	&
	2/100
	&
	\tableplot{2,98}
	&
	22 105 171
	&
	345.42
	&&
	\textbf{60*}
	&
	61.0
	&
	5/100
	&
	\tableplot{5,92,3}
	&
	40.4
	&
	18 905 823
	&
	154.84
	\\
	NRG.4
	&
	61
	&&
	66
	&
	0.02
	&&
	\textbf{60*}
	&
	61.0
	&
	5/100
	&
	\tableplot{5,94,1}
	&
	16 489 876
	&
	145.45
	&&
	\textbf{60*}
	&
	61.0
	&
	5/100
	&
	\tableplot{5,95}
	&
	39.8
	&
	19 101 112
	&
	154.76
	\\
	NRG.5
	&
	61
	&&
	66
	&
	0.03
	&&
	\textbf{60*}
	&
	61.0
	&
	3/100
	&
	\tableplot{3,97}
	&
	19 887 906
	&
	260.87
	&&
	\textbf{60*}
	&
	61.0
	&
	4/100
	&
	\tableplot{4,95,1}
	&
	45.5
	&
	22 879 725
	&
	108.45
	\\
	NRH.1
	&
	34
	&&
	36
	&
	0.01
	&&
	\textbf{33*}
	&
	34.0
	&
	2/100
	&
	\tableplot{2,98}
	&
	20 006 919
	&
	1310.66
	&&
	\textbf{33*}
	&
	34.0
	&
	3/100
	&
	\tableplot{3,97}
	&
	60.0
	&
	24 893 831
	&
	782.85
	\\
	NRH.2
	&
	34
	&&
	36
	&
	0.01
	&&
	\textbf{33*}
	&
	34.0
	&
	1/100
	&
	\tableplot{1,99}
	&
	19 185 701
	&
	1164.08
	&&
	\textbf{34}
	&
	34.0
	&
	100/100
	&
	&
	41.4
	&
	8 565 568
	&
	224.75
	\\
	NRH.3
	&
	34
	&&
	36
	&
	0.03
	&&
	\textbf{34}
	&
	34.0
	&
	100/100
	&
	&
	4 182 503
	&
	194.41
	&&
	\textbf{33*}
	&
	34.0
	&
	2/100
	&
	\tableplot{2,98}
	&
	66.0
	&
	18 405 339
	&
	563.87
	\\
	NRH.4
	&
	34
	&&
	36
	&
	0.02
	&&
	\textbf{33*}
	&
	34.0
	&
	2/100
	&
	\tableplot{2,98}
	&
	17 019 439
	&
	549.95
	&&
	\textbf{33*}
	&
	34.0
	&
	3/100
	&
	\tableplot{3,96,1}
	&
	65.0
	&
	20 008 033
	&
	618.99
	\\
	NRH.5
	&
	34
	&&
	36
	&
	0.01
	&&
	\textbf{33*}
	&
	34.0
	&
	1/100
	&
	\tableplot{1,99}
	&
	29 527 695
	&
	1853.87
	&&
	\textbf{33*}
	&
	34.0
	&
	1/100
	&
	\tableplot{1,99}
	&
	58.0
	&
	18 974 336
	&
	573.56
	\\
	CLR10
	&
	25
	&&
	33
	&
	0.00
	&&
	\textbf{25}
	&
	25.0
	&
	100/100
	&
	&
	322
	&
	0.02
	&&
	\textbf{25}
	&
	25.0
	&
	100/100
	&
	&
	0.0
	&
	314
	&
	0.09
	\\
	CLR11
	&
	23
	&&
	30
	&
	0.00
	&&
	\textbf{23}
	&
	23.0
	&
	100/100
	&
	&
	1 182
	&
	0.02
	&&
	\textbf{23}
	&
	23.0
	&
	100/100
	&
	&
	0.1
	&
	998
	&
	0.18
	\\
	CLR12
	&
	23
	&&
	32
	&
	0.00
	&&
	\textbf{23}
	&
	23.0
	&
	100/100
	&
	&
	3 009
	&
	0.06
	&&
	\textbf{23}
	&
	23.0
	&
	100/100
	&
	&
	0.2
	&
	2 483
	&
	0.33
	\\
	CLR13
	&
	23
	&&
	32
	&
	0.01
	&&
	\textbf{23}
	&
	23.0
	&
	100/100
	&
	&
	24 840
	&
	0.65
	&&
	\textbf{23}
	&
	23.0
	&
	100/100
	&
	&
	0.9
	&
	15 372
	&
	1.13
	\\
	CYC6
	&
	60
	&&
	\textbf{60}
	&
	0.00
	&&
	\textbf{60}
	&
	60.0
	&
	100/100
	&
	&
	0
	&
	0.00
	&&
	\textbf{60}
	&
	60.0
	&
	100/100
	&
	&
	1.1
	&
	1 587
	&
	0.51
	\\
	CYC7
	&
	144
	&&
	148
	&
	0.00
	&&
	\textbf{144}
	&
	144.0
	&
	100/100
	&
	&
	0
	&
	0.01
	&&
	\textbf{144}
	&
	144.0
	&
	100/100
	&
	&
	3.9
	&
	18 428
	&
	0.41
	\\
	CYC8
	&
	342
	&&
	364
	&
	0.02
	&&
	\textbf{342}
	&
	342.0
	&
	100/100
	&
	&
	166 179
	&
	0.56
	&&
	\textbf{342}
	&
	342.0
	&
	100/100
	&
	&
	13.6
	&
	401 988
	&
	2.36
	\\
	CYC9
	&
	772
	&&
	816
	&
	0.25
	&&
	\textbf{772}
	&
	773.9
	&
	43/100
	&
	\tableplot{43,0,35,0,3,12,5,0,2}
	&
	23 096 806
	&
	108.76
	&&
	\textbf{772}
	&
	773.5
	&
	64/100
	&
	\tableplot{64,0,16,0,0,4,16}
	&
	46.9
	&
	31 394 885
	&
	116.56
	\\
	CYC10
	&
	1798
	&&
	1928
	&
	2.84
	&&
	\textbf{1794*}
	&
	1798.6
	&
	1/100
	&
	\tableplot{1,0,7,0,55,0,37}
	&
	6 457 202
	&
	75.67
	&&
	\textbf{1792*}
	&
	1792.8
	&
	60/100
	&
	\tableplot{60,3,34,0,3}
	&
	32.3
	&
	31 499 213
	&
	249.49
	\\
	CYC11
	&
	3968
	&&
	4304
	&
	22.16
	&&
	\textbf{3968}
	&
	4011.3
	&
	4/100
	&
	\tableplot{4,0,0,0,0,0,0,0,0,0,0,0,0,0,0,0,0,3,1,5,7,3,5,0,0,1,0,1,0,1,0,1,0,0,0,0,2,3,2,3,4,3,3,2,5,0,2,2,0,0,1,0,2,1,1,1,0,1,1,1,1,4,1,5,2,1,0,3,0,1,1,0,0,0,0,0,2,1,0,0,2,0,1,0,0,0,0,0,0,0,0,0,1,0,0,0,1,0,0,0,0,0,1}
	&
	12 733 437
	&
	296.68
	&&
	\textbf{3968}
	&
	4004.8
	&
	3/100
	&
	\tableplot{3,0,0,0,0,0,0,0,0,0,0,0,0,0,0,0,0,0,0,1,1,0,2,0,1,0,3,3,1,1,1,1,4,4,2,1,4,4,9,4,9,6,11,7,5,7,3,0,0,2}
	&
	14.7
	&
	6 343 291
	&
	209.22
	\\
	RAIL507
	&
	96
	&&
	120
	&
	0.06
	&&
	97
	&
	97.0
	&
	100/100
	&
	&
	59 202 059
	&
	578.46
	&&
	\textbf{96}
	&
	97.0
	&
	1/100
	&
	\tableplot{1,99}
	&
	300.0
	&
	1 273 402 654
	&
	6797.13
	\\
	RAIL516
	&
	134
	&&
	151
	&
	0.08
	&&
	\textbf{134}
	&
	134.0
	&
	100/100
	&
	&
	189 885
	&
	2.05
	&&
	\textbf{134}
	&
	134.0
	&
	100/100
	&
	&
	1.9
	&
	324 399
	&
	5.02
	\\
	RAIL582
	&
	126
	&&
	150
	&
	0.12
	&&
	127
	&
	127.0
	&
	100/100
	&
	&
	72 058 325
	&
	681.93
	&&
	\textbf{126}
	&
	127.0
	&
	1/100
	&
	\tableplot{1,99}
	&
	278.0
	&
	1 028 182 991
	&
	5532.66
	\\
	RAIL2536
	&
	378
	&&
	506
	&
	26.85
	&&
	385
	&
	386.7
	&
	2/100
	&
	\tableplot{2,27,66,5}
	&
	14 434 039
	&
	4213.55
	&&
	385
	&
	387.0
	&
	1/100
	&
	\tableplot{1,12,70,17}
	&
	10.0
	&
	49 204 934
	&
	8055.58
	\\
	RAIL2586
	&
	518
	&&
	639
	&
	18.04
	&&
	529
	&
	531.7
	&
	2/100
	&
	\tableplot{2,6,26,51,15}
	&
	66 719 527
	&
	7464.89
	&&
	530
	&
	532.0
	&
	1/100
	&
	\tableplot{1,25,48,22,4}
	&
	18.0
	&
	78 656 704
	&
	6425.40
	\\
	RAIL4284
	&
	594
	&&
	802
	&
	67.68
	&&
	613
	&
	614.9
	&
	9/100
	&
	\tableplot{9,27,37,24,3}
	&
	35 537 195
	&
	6478.20
	&&
	612
	&
	615.4
	&
	2/100
	&
	\tableplot{2,5,11,35,27,17,3}
	&
	12.0
	&
	70 046 100
	&
	8770.11
	\\
	RAIL4872
	&
	879
	&&
	1101
	&
	83.62
	&&
	898
	&
	900.7
	&
	2/100
	&
	\tableplot{2,10,31,34,20,3}
	&
	73 783 226
	&
	6625.62
	&&
	897
	&
	900.7
	&
	1/100
	&
	\tableplot{1,4,11,24,37,19,4}
	&
	15.0
	&
	81 576 075
	&
	5293.67
	\\
	STS9
	&
	5
	&&
	\textbf{5}
	&
	0.00
	&&
	\textbf{5}
	&
	5.0
	&
	100/100
	&
	&
	0
	&
	0.00
	&&
	\textbf{5}
	&
	5.0
	&
	100/100
	&
	&
	0.0
	&
	0
	&
	0.03
	\\
	STS15
	&
	9
	&&
	\textbf{9}
	&
	0.00
	&&
	\textbf{9}
	&
	9.0
	&
	100/100
	&
	&
	0
	&
	0.00
	&&
	\textbf{9}
	&
	9.0
	&
	100/100
	&
	&
	0.0
	&
	0
	&
	0.02
	\\
	STS27
	&
	18
	&&
	19
	&
	0.00
	&&
	\textbf{18}
	&
	18.0
	&
	100/100
	&
	&
	6
	&
	0.00
	&&
	\textbf{18}
	&
	18.0
	&
	100/100
	&
	&
	0.0
	&
	6
	&
	0.02
	\\
	STS45
	&
	30
	&&
	33
	&
	0.00
	&&
	\textbf{30}
	&
	30.0
	&
	100/100
	&
	&
	3 016
	&
	0.01
	&&
	\textbf{30}
	&
	30.0
	&
	100/100
	&
	&
	1.5
	&
	1 958
	&
	0.12
	\\
	STS81
	&
	61
	&&
	65
	&
	0.00
	&&
	\textbf{61}
	&
	61.0
	&
	100/100
	&
	&
	90
	&
	0.01
	&&
	\textbf{61}
	&
	61.0
	&
	100/100
	&
	&
	0.0
	&
	147
	&
	0.03
	\\
	STS135
	&
	103
	&&
	111
	&
	0.00
	&&
	\textbf{103}
	&
	103.0
	&
	100/100
	&
	&
	700 927
	&
	1.78
	&&
	\textbf{103}
	&
	103.0
	&
	100/100
	&
	&
	7.3
	&
	651 729
	&
	2.47
	\\
	STS243
	&
	198
	&&
	211
	&
	0.01
	&&
	\textbf{198}
	&
	198.0
	&
	100/100
	&
	&
	13 110
	&
	0.11
	&&
	\textbf{198}
	&
	198.0
	&
	100/100
	&
	&
	0.3
	&
	9 808
	&
	0.18
	\\
	STS405
	&
	335
	&&
	357
	&
	0.04
	&&
	\textbf{335}
	&
	335.6
	&
	43/100
	&
	\tableplot{43,57}
	&
	24 899 317
	&
	180.63
	&&
	\textbf{335}
	&
	335.0
	&
	100/100
	&
	&
	43.5
	&
	10 859 761
	&
	67.22
	\\
	STS729
	&
	617
	&&
	665
	&
	0.63
	&&
	\textbf{617}
	&
	619.6
	&
	88/100
	&
	\tableplot{88,0,0,0,0,0,0,0,0,0,0,0,0,0,0,0,0,0,0,0,4,0,8}
	&
	1 994 273
	&
	43.36
	&&
	\textbf{617}
	&
	617.0
	&
	100/100
	&
	&
	1.6
	&
	466 653
	&
	13.74
	\\
	STS1215
	&
	1063
	&&
	1119
	&
	4.27
	&&
	\textbf{1063}
	&
	1065.3
	&
	3/100
	&
	\tableplot{3,0,77,1,19}
	&
	21 313 793
	&
	671.68
	&&
	\textbf{1063}
	&
	1065.9
	&
	2/100
	&
	\tableplot{2,0,53,0,42,3}
	&
	30.5
	&
	63 993 187
	&
	1598.55
	\\
	STS2187
	&
	1963
	&&
	2059
	&
	45.02
	&&
	\textbf{1963}
	&
	1967.6
	&
	81/100
	&
	\tableplot{81,0,0,0,0,0,0,0,0,0,0,0,0,0,0,0,1,0,0,0,2,1,0,0,11,0,2,0,0,0,0,0,2}
	&
	26 704 664
	&
	2118.68
	&&
	\textbf{1963}
	&
	1964.6
	&
	91/100
	&
	\tableplot{91,0,0,0,0,0,0,0,0,2,0,0,0,0,0,1,1,0,0,0,2,0,0,1,2}
	&
	9.7
	&
	43 003 667
	&
	3546.48
	\\
\end{longtable}

	\end{landscape}
\end{document}
